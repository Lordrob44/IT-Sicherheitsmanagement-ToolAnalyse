\subsection{Doc SetMinder}
Die Firma GRC-Partner GmbH bietet für ihre Softwarelösung keine Testversion an, da laut dem Hersteller eine mehrtägige Einarbeitung in das Tool in Form einer Schulung notwendig wäre, um sinnvoll damit arbeiten zu können. Es gab lediglich ein Angebot zur einstündigen Webpräsentation des Tools \cite{docsetminder}. 

\subsection{HiScout}
Ähnliches gilt auch für die HiScout GmbH: Trotz einer jahrelangen Zusammenarbeit mit verschiedenen Hochschulen, ist die Firma nicht bereit eine Testversion zu stellen. Das Angebot bestand in einer zeitlich befristeten Lizenzierung für die Lehre beziehungsweise eine unbefristete Lizenzierung für die Lehre sowie dem Einsatz an der Hochschule \cite{hiscout}.

\subsection{GRC Suite iRIS}
Die ibi Systems GmbH bietet ebenfalls keine Testversion an. Es wäre möglich gewesen, das Programm in einer der Niederlassungen in Regensburg persönlich vorgeführt zu bekommen. Dies wäre jedoch zeitlich sowie kostentechnisch nicht umsetzbar gewesen \cite{grcsuite}.

\subsection{opus i}
Nach einer zweistündigen Onlineeinführung in das Programm, stellte Herr Kron von der Kronsoft e.K. der Gruppe eine kostenlose Testversion zu Verfügung. Diese bot nur sehr wenige Einschränkungen, sodass das Tool sehr umfangreich getestet werden konnte. Während der Evaluation traten keine Abstürze oder unerklärliche Fehlermeldungen auf. Als Datengrundlage dient eine Datenbank, die auf einem Server gehostet von mehreren gleichzeitig bearbeitet werden kann. Somit ist ein verteiltes arbeiten problemlos möglich. Gleichzeitig können jedoch Rechte vergeben werden, sodass nicht für jeden alles ersichtlich oder änderbar ist. 
\\
\\
Das Programm überzeugt durch eine aufgeräumte Oberfläche, die in die verschiedenen Arbeitsbereiche unterteilt ist [\ref{opusiimage}]. In der linken unteren Ecke wird das gerade bearbeitete IT-Verbundsystem angezeigt, sodass man jederzeit einen Überblick über das System hat. Wizards beinhaltet das Programm keine, jedoch sind viele Hinweise, wie etwas eingestellt werden soll, an den entscheidenden Stellen gegeben. Änderungen können auch sehr bequem über die mittlere Arbeitsfläche eingepflegt werden.
\\
\\
Um Mehraufwand zu vermeiden, ist es möglich, Objekte in ein Verbundsystem zu referenzieren, das heißt, dass sobald Änderungen an diesem Objekt getätigt werden, sich diese global auf alle anderen Objekte übertragen. Die BSI-Standards sind laut dem Hersteller immer aktuell und Änderungen werden auch im Programm direkt abgebildet. Exporte von Berichten können sehr individuell gestaltet werden und unterstützen auch die Zertifizierung. Importe vom GS-Tool seien ebenfalls problemlos möglich.
\\
\\
Bevor die eigentlichen Schutzmaßnahmen festgelegt werden, kann eine Risikoanalyse für das Verbundsystem erfolgen. Dadurch ist es möglich, dass das Tool eine Unterscheidung zwischen wichtigen und weniger wichtigen Maßnahmen geben kann. Die Dokumentation für das Programm ist sehr gut umgesetzt. Herr Kron hat zudem ein bereits im Jahr 2012 unterbreitetes Angebot wieder vorgelegt, bei dem 20 Lizenzen zu einem Preis von 3200 Euro zu erhalten wären mit einem Wartungsvertrag über 256 Euro pro Jahr. \cite{opusi}.
\\
\\
Bewertung:
\begin{itemize}
\item \textbf{Wizard:} Keine Wizards zur Projekterstellung vorhanden, jedoch aufgeräumte Oberfläche und viele nützliche Tipps. \textbf{Punkte: 9}
\item \textbf{Infrastrukturdarstellung:} Darstellung als Baumstruktur. Sehr übersichtlich gestaltet und in der unteren linken Ecke immer in Sichtweite. \textbf{Punkte: 9}
\item \textbf{Netzpläne, Prozessflüsse:} Sind in dem Programm nicht enthalten. \textbf{Punkte: 0}
\item \textbf{Schnelles Einpflegen von Änderungen: } Änderungen können einfach und schnell über die Arbeitsfläche eingebracht werden.  \textbf{Punkte: 9}
\item \textbf{Doppelseitige Verlinkungen:} In dem Programm so nicht enthalten, jedoch sehr freie Gestaltung bei der Modellierung des IT-Verbunds.  \textbf{Punkte: 4}
\item \textbf{Vererbung:} Objekte können als Referenzen eingebunden werden. Damit sind Änderungen in einem IT-Verbund auch für alle anderen gültig.  \textbf{Punkte: 10}
\item \textbf{Gruppierungen:} Gruppierungen sind möglich. Änderungen können für Gruppenelemente übergreifend durchgeführt werden.  \textbf{Punkte: 10}
\item \textbf{Aktuelle BSI-Standards:} Änderungen am BSI-Standard werden zeitnah eingepflegt und auch im Programm selbst angezeigt.  \textbf{Punkte: 10}
\item \textbf{Erweiterbarkeit der Klassifizierungen:} Elemente können angelegt oder erweitert werden. Auch eigene Skriptsprache vorhanden.  \textbf{Punkte: 9}
\item \textbf{Individuelle Beschreibungen:} Elemente können nach belieben beschrieben und modelliert werden.  \textbf{Punkte: 10}
\item \textbf{Sicherheitsverstöße markieren:} In diesem Programm nicht enthalten.  \textbf{Punkte: 0}
\item \textbf{Export von Berichten:} Große Anzahl von Exportmöglichkeiten. Individuell anpassbar.  \textbf{Punkte: 10}
\item \textbf{Bewertung des Sicherheitsstatus:} Auswertung des Fortschritts der umgesetzten Maßnahmen möglich.  \textbf{Punkte: 9}
\item \textbf{GS-Tool import:} Daten aus dem GS-Tool können ohne Probleme importiert werden.  \textbf{Punkte: 10}
\item \textbf{Sicherheit des Tools:} Passwortgeschützt. Jedoch sind alle Daten in der Datenbank gespeichert. Sollte diese in fremde Hände gelangen, könnten vertrauliche Daten ausgelesen werden.  \textbf{Punkte: 5}
\item \textbf{Risikobewertung:} Durchführen einer Risikoanalyse für das IT-Verbundsystem möglich. Damit kann eine Vorabbeurteilung der kritischen und weniger kritischen Maßnahmen getroffen werden. AttackTrees aber fehlen. \textbf{Punkte: 8}
\item \textbf{Verteiltes Arbeiten:} Gleichzeitiger und verteilter Zugriff auf Datenbank, die auf einem Server gehostet wird, möglich. \textbf{Punkte: 10}
\item \textbf{Rechtevergabe:} Rechtevergabe enthalten. Verschiedene Zugriffsstufen enthalten. \textbf{Punkte: 10}
\item \textbf{Kosten:} 20 Lizenzen für 3200 Euro sowie ein Wartungsvertrag von 256 Euro pro Monat. \textbf{Punkte: 10}
\item \textbf{Zertifizierung:} Exporte von Berichten, die zur Zertifizierung notwendig sind, können erstellt werden. \textbf{Punkte: 10}
\item \textbf{Support:} Schnelle Antworten auf Emails, freundlicher Kontakt, Hilfsbereitschaft. \textbf{Punkte: 10}
\item \textbf{Dokumentation:} Umfangreiche Dokumentation aller Funktionalitäten im Programm enthalten. \textbf{Punkte: 10}
\item \textbf{Marktpräsenz:} Namenhafte Unternehmen wie IKEA, T-Mobile oder Sparkasse Leipzig sind Anwender des Programms. \textbf{Punkte: 8}
\item \textbf{Nutzerkreis/Zielgruppe:} Eher an Firmen und größere Unternehmen gerichtet. \textbf{Punkte: 8}
\item \textbf{Pflege/Weiterentwicklung:} Stetige Entwicklung und regelmäßige Updates (BSI-Standards). \textbf{Punkte: 10}

\end{itemize}

Aufgrund des umfangreichen Funktionsangebots, das vor allem darauf abzielt, den Anwender möglichst von zu vielen Nebentätigkeiten bei der Schutzbedarfsanalyse und deren Umsetzung zu entlasten sowie dem relativ günstigen Preis und dem guten Support, ist \textit{opus i} eine sehr gute Alternative zum GS-Tool.

\begin{table}[h!tb]
	%\centering
	\begin{tabular}{|p{0.5\textwidth}|p{0.5\textwidth}|}
		\hline 
		Kriterium & Bewertung\\ 
		\hline 
		\textbf{GUI}& \\
		\hline
		Wizard & 9\\
		\hline 
		Infrastrukturdarstellung & 9 \\
		\hline 
		Netzpläne & 0 \\
		\hline 
		Prozessflüsse & 0 \\
		\hline 
		Schnelles Einpflegen von Änderungen & 9 \\
		\hline
		\textbf{Objektrelationen} & \\
		\hline 
		Doppelseitige Verlinkungen & 4 \\
		\hline 
		Vererbung & 10 \\
		\hline 
		Gruppierungen & 10 \\
		\hline 
		\textbf{Funktionalität} &\\
		\hline 
		Aktuelle BSI-Standards & 10 \\
		\hline  
		Erweiterbarkeit der Klassifizierungen & 9 \\
		\hline 
		Individuelle Beschreibungen & 10 \\
		\hline 
		Sicherheitsverstöße markieren & 0 \\
		\hline
		Bewertung des Sicherheitsstatus & 9 \\
		\hline
		Export von Berichten & 10 \\
		\hline
		GS-Tool import & 10 \\
		\hline
		Sicherheit des Tools an sich & 5 \\
		\hline
		Risikobewertung & 8 \\
		\hline
		\textbf{System}&  \\
		\hline
		Verteiltes Arbeiten & 10 \\
		\hline
		Rechtevergabe & 10 \\
		\hline
		Kosten & 10 \\
		\hline
		Support & 10 \\
		\hline
		Zertifizierung & 10 \\
		\hline
		Dokumentation & 10 \\
		\hline
		Marktpräsenz & 8 \\
		\hline
		Spezielle Zielgruppe & 8 \\
		\hline
		Pflege/Weiterentwicklung & 10 \\
		\hline
		\multicolumn{2}{c}{}\\
		\hline
		\textbf{Gesamt} & \\
		\hline
		Hochschuleinsatz & 77\%\\
		\hline
		Lehre & 80\%\\
		\hline
	\end{tabular} 
	\caption{Berwertung: opus i}
	\label{tab:Berwertungopusi}
\end{table}
