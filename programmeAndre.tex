\subsection{Doc SetMinder}
Die Firma GRC-Partner GmbH bietet für ihre Softwarelösung keine Testversion an, da laut dem Hersteller eine mehrtägige Einarbeitung in das Tool in Form einer Schulung notwendig wäre, um sinnvoll damit arbeiten zu können. Es gab lediglich ein Angebot zur einstündigen Webpräsentation des Tools \cite{docsetminder}. 

\subsection{HiScout}
Ähnliches gilt auch für die HiScout GmbH: Trotz einer jahrelangen Zusammenarbeit mit verschiedenen Hochschulen, ist die Firma nicht bereit eine Testversion zu stellen. Das Angebot bestand in einer zeitlich befristeten Lizenzierung für die Lehre beziehungsweise eine unbefristete Lizenzierung für die Lehre sowie dem Einsatz an der Hochschule \cite{hiscout}.

\subsection{GRC Suite iRIS}
Die ibi Systems GmbH bietet ebenfalls keine Testversion an. Es wäre möglich gewesen, das Programm in einer der Niederlassungen in Regensburg persönlich vorgeführt zu bekommen. Dies wäre jedoch zeitlich sowie kostentechnisch nicht umsetzbar gewesen \cite{grcsuite}.

\subsection{opus i}
Nach einer zweistündigen Onlineeinführung in das Programm, stellte Herr Kron von der Kronsoft e.K. der Gruppe eine kostenlose Testversion zu Verfügung. Diese bot nur sehr wenige Einschränkungen, sodass das Tool sehr umfangreich getestet werden konnte. Während der Evaluation traten keine Abstürze oder unerklärliche Fehlermeldungen auf. Als Datengrundlage dient eine Datenbank, die auf einem Server gehostet von mehreren gleichzeitig bearbeitet werden kann. Somit ist ein verteiltes arbeiten problemlos möglich. Gleichzeitig können jedoch Rechte vergeben werden, sodass nicht für jeden alles ersichtlich oder änderbar ist. 
\\
\\
Das Programm überzeugt durch eine aufgeräumte Oberfläche, die in die verschiedenen Arbeitsbereiche unterteilt ist. In der linken unteren Ecke wird das gerade bearbeitete IT-Verbundsystem angezeigt, sodass man jederzeit einen Überblick über das System hat. Wizards beinhaltet das Programm keine, jedoch sind viele Hinweise, wie etwas eingestellt werden soll, an den entscheidenden Stellen gegeben. Änderungen können auch sehr bequem über die mittlere Arbeitsfläche eingepflegt werden.
\\
\\
Um Mehraufwand zu vermeiden, ist es möglich, Objekte in ein Verbundsystem zu referenzieren, das heißt, dass sobald Änderungen an diesem Objekt getätigt werden, sich diese global auf alle anderen Objekte übertragen. Die BSI-Standards sind laut dem Hersteller immer aktuell und Änderungen werden auch im Programm direkt abgebildet. Exporte von Berichten können sehr individuell gestaltet werden und unterstützen auch die Zertifizierung. Importe vom GS-Tool seien ebenfalls problemlos möglich.
\\
\\
Bevor die eigentlichen Schutzmaßnahmen festgelegt werden, kann eine Risikoanalyse für das Verbundsystem erfolgen. Dadurch ist es möglich, dass das Tool eine Unterscheidung zwischen wichtigen und weniger wichtigen Maßnahmen geben kann. Die Dokumentation für das Programm ist sehr gut umgesetzt. Herr Kron hat zudem ein bereits im Jahr 2012 unterbreitetes Angebot wieder vorgelegt, bei dem 20 Lizenzen zu einem Preis von 3200 Euro zu erhalten wären mit einem Wartungsvertrag über 256 Euro pro Jahr. Aufgrund des umfangreichen Funktionsangebots, das vor allem darauf abzielt, den Anwender möglichst von zu vielen Nebentätigkeiten bei der Schutzbedarfsanalyse und deren Umsetzung zu entlasten sowie dem relativ günstigen Preis und dem guten Support, ist opus i eine sehr gute Alternative zum GS-Tool \cite{opusi}.