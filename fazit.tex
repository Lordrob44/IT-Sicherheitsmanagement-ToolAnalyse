\section{Fazit}
Aufgabe war es, einen Ersatz für das derzeit im Einsatz befindliche BSI-GS-Tool zu finden, sowohl für den Einsatz in der Lehre als auch zur Realisierung der IT-Sicherheit an der Hochschule. Untersucht wurden dazu die vom BSI empfohlenen Programme. Die Ergebnisse dieser Untersuchung sind Tabelle~ref{programmvergleich} zu entnehmen.

Als erstes ist hervorstechend, dass einige der empfohlenen Programme nicht mehr gepflegt werden und somit für eine Nutzung in beiden Bereichen ausscheiden. Auch das \textit{BSI-GS-Tool}, welches als Vergleich dient, hat mit den von uns aufgestellten Kriterien nur ca. 70\% der maximalen Punktzahl erreicht. Die höchste Punktzahl, knapp vor \textit{opus i}, erreicht \textit{I-doit} mit 83\% und ist somit unsere Empfehlung als Ersatz für das BSI-GS-Tool. Trotzdem hat es seine Unzulänglichkeiten in einige Bereichen. Abschließend ist zu erwähnen, dass auch wenn nicht alle Programme gute Gesamtwertungen erreicht haben, können sie für machen Ansprüche trotzdem genügen, was im Einzelfall zu prüfen ist. Denn wie eingangs geschrieben zählt die Umsetzung im Unternehmen, dazu ist mit gewissen Abstrichen jedes der testbaren Programme zumindest fähig.
\begin{table}[h!tb]
	%\centering
	\begin{tabular}{|p{0.5\textwidth}|p{0.2\textwidth}|p{0.2\textwidth}|}
		\hline 
		\textbf{Programm} & \textbf{Hochschule} & \textbf{Lehre}\\ 
		\hline
		Sidoc & 52\% & 55\% \\
		\hline 
		Doc SetMinder & \multicolumn{2}{c|}{keine Testversion; zu hohe Einarbeitung}\\
		\hline 
		HiScout & \multicolumn{2}{c|}{Angebot, aber keine freie Testversion} \\
		\hline 
		GRC Suite iRIS & \multicolumn{2}{c|}{keine Testversion; organisatorisch nicht machbar}\\
		\hline 
		opus i & 77\% & 80\% \\
		\hline
		INDART Professional & \multicolumn{2}{c|}{keine funktionierende Testversion} \\
		\hline 
		DHC VISION & \multicolumn{2}{c|}{organisatorisch nicht machbar} \\
		\hline 
		SAVe - Security Audit and Verification & 33\% & 31\% \\
		\hline 
		Adamant & 35\% & 43\%\\
		\hline 
		GS-Tool & 67\% & 68\% \\
		\hline  
		I-doit & 83\% & 83\% \\
		\hline
		Audit-Tool 2009 & \multicolumn{2}{c|}{keine Testversion} \\
		\hline
		Verinice & 72\% & 64\% \\
		\hline
		Crisam & 71\% & 70\% \\
		\hline
	\end{tabular} 
	\caption{Vergleich der getesteten Programme}
	\label{tab:programmvergleich}
\end{table}
