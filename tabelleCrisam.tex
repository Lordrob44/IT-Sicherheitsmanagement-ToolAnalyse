
\documentclass{book}

\usepackage[ngerman]{babel}
\usepackage[utf8]{inputenc}
\usepackage[T1]{fontenc}
\usepackage{amsmath}
\usepackage{tabularx}
\usepackage{graphicx}
\usepackage{listings}
\usepackage{natbib}
\usepackage[top=2cm,bottom=2cm,left=2cm,right=2cm]{geometry}

\begin{document}

\subsection{Crisam}
\begin{table}[h]
%\centering
\begin{tabular}{|p{0.4\textwidth}|p{0.15\textwidth}|p{0.15\textwidth}|p{0.15\textwidth}|}
\hline 
Kriterium & Bewertung & Hochschule & Lehre\\ 
\hline 
\textbf{GUI}& & &\\
\hline
Wizard & 7 & 21 & 28\\
\hline 
Infrastrukturdarstellung & 10 & 80 & 80 \\
\hline 
Netzpläne & 3 & 15 & 15\\
\hline 
Prozessflüsse & 5 & 25 & 25\\
\hline 
Schnelles Einpflegen von Änderungen & 7 & 56 & 56\\
\hline
\textbf{Objektrelationen} & & & \\
\hline 
Doppelseitige Verlinkungen & 0 & 0 & 0 \\
\hline 
Vererbung & 10 & 90 & 90\\
\hline 
Gruppierungen & 8 & 56 & 16 \\
\hline 
\textbf{Funktionalität} & & &\\
\hline 
Aktuelle BSI-Standards & 10 & 100 & 100 \\
\hline  
Erweiterbarkeit der Klassifizierungen & 10 & 70 & 70\\
\hline 
Individuelle Beschreibungen & 0 & 0 & 0\\
\hline 
Sicherheitsverstöße markieren & 8 & 72 & 56 \\
\hline
Bewertung des Sicherheitsstatus & 10 & 100 & 100 \\
\hline
Export von Berichten & 0 & 0 & 0 \\
\hline
BSI-Toolimport & 10 & 10 & 0\\
\hline
Sicherheit des Tools an sich & 10 & 100 & 0\\
\hline
Risikobewertung & 0 & 0 & 0\\
\hline
\textbf{System}& & &\\
\hline
Verteiltes Arbeiten & 10 & 20 & 0 \\
\hline
Rechtevergabe & 10 & 70 & 0 \\
\hline
Kosten & 10 & 90 & 100\\
\hline
Support & 8 & 56 & 24 \\
\hline
Zertifizierung & 10 & 100 & 0\\
\hline
Dokumentation & 5 & 45 & 50 \\
\hline
Marktpräsenz & 10 & 70 & 40 \\
\hline
Spezielle Zielgruppe & 4 & 16 & 16 \\
\hline
Pflege/Weiterentwicklung & 0 & 0 & 0\\
\hline
\multicolumn{4}{c}{}{}{}\\
\hline
\textbf{Gesamt} & {} & {Hochschule} & {Lehre}\\
% & \\
\hline
Summe & & 1342 & 854\\
\hline
Prozent & & 71,28 & 70,58 \\
\hline
\end{tabular} 
\caption{Berwertung: Crisam}
\label{tab:Berwertung Crisam}
\end{table}
Die Auswertungstabelle zeigt die von mir bewerteten Kriterien und die daraus resultieren Daten für die Hochschule und die Lehre. 
Die Kriterien wurden untergliedert in Gruppen nach GUI, Objekrelationen, Funktionalität und System.
\\
Bei GUI wurden die Bewertung von 3 bis 10 Punkte vorgenommen. 
Die niedrigste Punktvergabe hat Netzpläne, weil sie im Programm nicht ausführlich vorkommen.
\\
Die Infrastrukturdarstellung wurde mit 10 Punkte bewertet, weil verschiedene Darstellungen möglich sind.
In der Gruppe der Objektrelationen wurde die Vererbung mit 10 Punkten bewertet, da sie sehr gut vorhanden sind.
\\
In der Funktionalitätsgruppe sind aktuelle BSI-Standards, Erweiterbarkeit der Klassifizierungen, Bewertung des Sicherheitsstatus, BSI-Toolimport und Sicherheit des Tools mit 10 Punkten bewertet, weil sie im Programm sichtbar sind.
Bei der letzten Gruppenabteilung sind die Dokumentation und spezielle Zielgruppen am schlechtesten bewertet, da Tutorials und Dokumentation zur Nutzung des Tool nicht vorhanden waren. Deshalb ist das Tool nicht für die Ausbildung und zum Zweck des Studiums geeignet, weil man es nicht umfangreich testen konnte.
\\
\\
Die Ergebnisse zeigen sowohl für die Hochschule als auch für die Lehre etwa 70 Prozent, das bedeutet das Crisam ein geeignete GS-Tool ist.

\end{document}