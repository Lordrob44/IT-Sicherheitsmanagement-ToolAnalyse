\section{Bewertungskriterien}
Um alle untersuchten Programme einheitlich miteinander vergleichen zu können, wurde eine Liste von Bewertungskriterien erarbeitet, welche im folgenden kurz erklärt werden soll.\\
Die Bewertungsskala reicht von 0 bis 10, in aufsteigender Wertigkeit. Soweit nicht gesondert erwähnt, wird ein Merkmal mit 0 gewertet, wenn es nicht vorhanden ist und mit 10, wenn es wirklich gut ausgearbeitet ist.\\
Um die abschließende Wertung für den Einsatz in der Hochschule bzw. in der Lehre zu ermitteln, werden die Rohwertungen in einer Kategorie mit einer Gewichtung multipliziert, welche Tabelle ~\ref{tab:gewichtung} zu entnehmen ist. Die Gesamtwertung wird dann prozentual von der maximal möglichen Punktzahl angegeben.
\begin{itemize}
	\item \textit{Wizard} bewertet, ob es eine Führung durch das Programm mittels Wizards gibt und wie gut diese umgesetzt ist. 10 = gut umgesetzte benutzerfreundliche Wizardführung durch alle wichtigen Programmaspekte.

	\item \textit{Infrastrukturdarstellung} bewertet, wie umfangreich und übersichtlich die modellierten Komponenten Dargestellt werden.

	\item \textit{Netzpläne} bewertet, ob es Netzpläne gibt, welche die physischen sowie logischen Verbindungen abbilden.

	\item \textit{Prozessflüsse} bewertet, ob es Prozessgraphen gibt, welche u.A. die Informationsflüsse zwischen den einzelnen Komponenten darstellen.
	\item \textit{Schnelles Einpflegen von Änderungen} bewertet, wie einfach und unkompliziert schon erschaffene Komponenten modifiziert werden können.

	\item \textit{Doppelseitige Verlinkungen} bewertet, in welchem Maße eine Unterstützung zur Verlinkung von zwei Komponenten in beiden Richtungen ermöglicht wird. 10 = automatische doppelseitige Verlinkung, mit Möglichkeit zur Auflösung

	\item \textit{Vererbung}  bewertet, ob miteinander verbundenen Komponenten entsprechende Eigenschaften erben. 10 = vollständige Vererbung, mit Option sie auch zu unterlassen

	\item \textit{Gruppierungen} bewertet, ob und wie komfortabel Gruppen gleichartiger Objekte zusammengefasst werden können. Bsp.: 10 gleiche PCs in einem PC-Pool

	\item \textit{Aktuelle BSI-Standards} bewertet, ob aktuelle BSI-Standards unterstützt werden. 10 = volle BSI-Unterstützung mit Auswahl von alternativen, d.h. man ist nicht auf im Programm festgeschriebene Kataloge beschränkt.

	\item \textit{Erweiterbarkeit der Klassifizierungen} bewertet, ob zusätzlich zu den im Programm vorhandenen Klassifizierungen auch eigene erstellt werden können. Bsp.: zusätzlich zum existierenden WLAN ein LAN erstellen. 10 = Möglichkeit vorhandene Klassifizierungen als Vorlage verwendbar, mit entsprechend verbundenen Eigenschaften.

	\item \textit{Individuelle Beschreibungen} bewertet, wie gut beschreibende bzw. ergänzende Texte eingebunden werden können.

	\item \textit{Sicherheitsverstöße markieren} bewertet, ob es eine grafische Rückmeldung bei gefundene Verstößen gibt und wie gut diese ausgeführt ist.
	\item \textit{Bewertung des Sicherheitsstatus} bewertet, ob und wie übersichtlich eine Bewertung des Sicherheitsstatus erfolgt.
	\item \textit{Export von Berichten} bewertet, die Möglichkeit Berichte zu exportieren. Verschiedene Ausgabeformate, Auswahl der darzustellenden Informationen und übersichtliche Darstellung verbessern die Wertung.
	\item \textit{BSI-Toolimport} bewertet, ob und wie gut es Möglich ist vorhandene Projekte aus dem BSI-GS-Tool zu importieren.
	\item \textit{Sicherheit des Tools an sich} bewertet, u.A. die Art der Speicherung der Daten(z.B. Klartext, verschlüsselt, ...).
	\item \textit{Risikobewertung} bewertet, wie gut und übersichtlich die Bewertung des Risikos durch das Programm realisiert wird. 
	\item \textit{Verteiltes Arbeiten} bewertet, ob das Programm von sich aus verteiltes Arbeiten unterstützt. Damit ist nicht die Ablage der Daten in einem Repository o.Ä. gemeint, sondern programminterne Optionen.
	\item \textit{Rechtevergabe} bewertet, ob der Zugriff/die Bearbeitung im Programm durch Nutzerrechte geschützt ist. 
	\item \textit{Kosten} bewertet, wie Teuer eine Einzellizenz im Jahr ist. Bezugsmaß ist das GS-Tool mit ca. 400 Euro pro Jahr. >5 = günstiger als GS-Tool, 10 = kostenlos 
	\item \textit{Support} bewertet, wie freundlich und gut erreichbar der Kundensupport ist. Bonus für kostenlose Hilfe, verschiedene Kontaktmöglichkeiten und schnelle Antwortzeiten.
	\item \textit{Zertifizierung} bewertet, ob mit den Ausgaben des Programms eine Zertifizierung nach ISO 27001 möglich ist. 0 = nein, 10 = ja.
	\item \textit{Dokumentation} bewertet, wie umfangreich und vor allem hilfreich die vorhandene Programmdokumentation ist. 10 = vollständige Programmdokumentation mit tiefgreifenden, verständlichen Tutorials.
	\item \textit{Marktpräsenz} bewertet, ob und in welchem Maße das Tool bereits verwendet wird. 10 = min. ein namhaftes Unternehmen oder zahlreiche kleinere mit positivem Feedback. Abzüge bzw. evtl. 0-Wertung für Negativbeispiele.
	\item \textit{Spezielle Zielgruppe} bewertet, ob das Tool eine spezielle Zielgruppe hat, bzw. ob es an die Bedürfnisse gewisser Zielgruppen anpassbar ist. Denkbar wären Spezialisierungen auf die Lehre, Betriebswirtschaft, etc.
	\item \textit{Pflege/Weiterentwicklung} bewertet, ob das Tool noch weiterentwickelt bzw. gepflegt wird.0 = Weiterentwicklung eingestellt, 10 = ja, häufiger, als es Änderungen an den GS-Materialien gibt.

\begin{table}[h!tb]
	%\centering
	\begin{tabular}{|p{0.5\textwidth}|p{0.2\textwidth}|p{0.2\textwidth}|}
		\hline 
		Kriterium & Hochschule & Lehre\\ 
		\hline 
		\textbf{GUI}& &\\
		\hline
		Wizard & 3 & 4\\
		\hline 
		Infrastrukturdarstellung & 8 & 8 \\
		\hline 
		Netzpläne & 5 & 5 \\
		\hline 
		Prozessflüsse & 5 & 5 \\
		\hline 
		Schnelles Einpflegen von Änderungen & 8 & 8 \\
		\hline
		\textbf{Objektrelationen} & &\\
		\hline 
		Doppelseitige Verlinkungen & 5 & 2 \\
		\hline 
		Vererbung & 9 & 9 \\
		\hline 
		Gruppierungen & 7 & 2 \\
		\hline 
		\textbf{Funktionalität} & & \\
		\hline 
		Aktuelle BSI-Standards & 10 & 10\\
		\hline  
		Erweiterbarkeit der Klassifizierungen & 7 & 7 \\
		\hline 
		Individuelle Beschreibungen & 8 & 4 \\
		\hline 
		Sicherheitsverstöße markieren & 9 & 7 \\
		\hline
		Bewertung des Sicherheitsstatus & 10 & 8 \\
		\hline
		Export von Berichten & 8 & 0 \\
		\hline
		GS-Tool import & 1 & 0 \\
		\hline
		Sicherheit des Tools an sich & 10 & 0\\
		\hline
		Risikobewertung & 10 & 1 \\
		\hline
		\textbf{System}& & \\
		\hline
		Verteiltes Arbeiten & 2 & 0 \\
		\hline
		Rechtevergabe & 7 & 0 \\
		\hline
		Kosten & 9 & 10\\
		\hline
		Support & 7 & 3 \\
		\hline
		Zertifizierung & 10 & 0 \\
		\hline
		Dokumentation & 9 & 10 \\
		\hline
		Marktpräsenz & 7 & 4 \\
		\hline
		Spezielle Zielgruppe & 4 & 4 \\
		\hline
		Pflege/Weiterentwicklung & 10 & 10 \\
		\hline
		
	\end{tabular} 
	\caption{Gewichtung nach Einsatz}
	\label{tab:gewichtung}
\end{table}
\end{itemize}
