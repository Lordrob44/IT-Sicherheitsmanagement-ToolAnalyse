\section{Einleitung}
Der IT-Grundschutz existiert seit 20 Jahren. Im Rahmen dieses Jubiläums hat das BSI beschlossen eine Optimierung und Aktualisierung der Vorgehensweise und IT-Grundschutz-Kataloge durchzuführen\cite{bsi}. Das BSI setzt sich die folgenden Ziele. Man will dem Bedarf der Anwender mit einem stets aktuellen und praxisnahen Verfahren gerecht werden. Dazu ist es auch wichtig, die Kontinuität zu gewährleisten und die alte IT-Grundschutz Welt weiterzuentwickeln. Das Ziel ist es hauptsächlich die Attraktivität zu erhöhen und den Weg für die nächsten 20 Jahre zu bereiten.

Die Toolunterstützung wird dabei als unabdingbar angesehen. Ein Tool kann dabei auch ein Wiki oder Ähnliches sein. Das Wichtigste ist eine gute Akzeptanz im Unternehmen, damit das Tool nicht ungenutzt bleibt. Es kommt alles in Frage, was die Arbeit mit dem IT-Grundschutz vor, während und nach einer Zertifizierung erleichtert und unterstützt.

\subsection{Aufgabenstellung}
An der Fachhochschule Görlitz wird für Lehrveranstaltungen zur Informationssicherheit und zum
Informationssicherheitsmanagement zurzeit das GSTOOL des BSI eingesetzt. Da die Weiterentwicklung dieses Tools eingestellt wurde und es in der Lehre wichtig ist aktuell zu bleiben, ist nach einer Alternative für den Einsatz in der Lehre zu suchen. 
Weiterhin ist es an der Hochschule dringend nötig, ein IT-Sicherheitskonzept zu erstellen und umzusetzen. Daher ist auch die Frage nach einem dafür geeigneten Tool zu klären. 

Dieses Projekt hat das Ziel Bewertungskriterien, sowohl für die Lehre, als auch den produktiven Einsatz an der Hochschule, auszuarbeiten, die von einem geeigneten Tool erfüllt werden müssen. Diese Bewertungskriterien sind auf die auf der Webseite des BSI genannten Alternativtools anzuwenden um einen Entscheidungsvorschlag zu erarbeiten.