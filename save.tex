\subsection{SAVe - Security Audit and Verification}
\begin{table}[h!tb]
	%\centering
	\begin{tabular}{|p{0.5\textwidth}|p{0.5\textwidth}|}
		\hline 
		Kriterium & Bewertung\\ 
		\hline 
		\textbf{GUI}& \\
		\hline
		Wizard & 6\\
		\hline 
		Infrastrukturdarstellung & 0 \\
		\hline 
		Netzpläne & 0 \\
		\hline 
		Prozessflüsse & 0 \\
		\hline 
		Schnelles Einpflegen von Änderungen & 10 \\
		\hline
		\textbf{Objektrelationen} & \\
		\hline 
		Doppelseitige Verlinkungen & 5 \\
		\hline 
		Vererbung & 3 \\
		\hline 
		Gruppierungen & 3 \\
		\hline 
		\textbf{Funktionalität} &\\
		\hline 
		Aktuelle BSI-Standards & 2 \\
		\hline  
		Erweiterbarkeit der Klassifizierungen & 2 \\
		\hline 
		Individuelle Beschreibungen & 8 \\
		\hline 
		Sicherheitsverstöße markieren & 2 \\
		\hline
		Bewertung des Sicherheitsstatus & 0 \\
		\hline
		Export von Berichten & 2 \\
		\hline
		BSI-Toolimport & 0 \\
		\hline
		Sicherheit des Tools an sich & 7 \\
		\hline
		Risikobewertung & 0 \\
		\hline
		\textbf{System}&  \\
		\hline
		Verteiltes Arbeiten & 1 \\
		\hline
		Rechtevergabe & 10 \\
		\hline
		Kosten & 4 \\
		\hline
		Support & 3 \\
		\hline
		Zertifizierung & 0 \\
		\hline
		Dokumentation & 2 \\
		\hline
		Marktpräsenz & 4 \\
		\hline
		Spezielle Zielgruppe & 1 \\
		\hline
		Pflege/Weiterentwicklung & 7 \\
		\hline
		\multicolumn{2}{c}{}\\
		\hline
		\textbf{Gesamt} & \\
		\hline
		Hochschuleinsatz & 33\%\\
		\hline
		Lehre & 31\%\\
		\hline
	\end{tabular} 
	\caption{Berwertung: SAVe}
	\label{tab:BerwertungSave}
\end{table}
SAVe\footnote{\url{http://www.infodas.de/DE/Produkte/SAVe}} ist ein weiteres Tool, welches vom BSI als Ersatz für das Grunschutztool empfohlen wird. Im Vergleich zum Grundschutztool kostet SAVe für 10 Lizenzen 7780,00 Euro plus 1167,00 Euro pro Jahr. Voraussetzung für das Programm ist Microsoft Access, welches zusätzlich gekauft werden muss. Für 10 Lizenzen der aktuellen Version von Microsoft Access sind 1350,00 Euro fällig. Damit wären für das erste Jahr 10297,00 Euro zu zahlen.\\
Für die Evaluation des Tools stand eine kostenlose 30-Tage Testversion zur Verfügung, die laut Hersteller keine Einschränkungen besitzt. Möchte man die kostenlose Testversion vom Hersteller bestellen, so kostet dies 49,99 Euro.\\
\\
Die Bewertung aus Tabelle \ref{tab:BerwertungSave} wird im Folgenden näher erläutert:\\
Eine Dokumentation ist sehr wichtig nicht nur für den Einsatz in der Lehre, sondern auch für den Einsatz in einer Firma. Es war jedoch nicht möglich diese Aufzurufen, obwohl sie im Programm enthalten ist. In diesem Fall kann man entweder auf Kulanz des Herstellers hoffen oder man benötigt einen gültigen Wartungsvertrag, da nur dieser die Reparatur defekter Programmversionen umfasst. Die fehlende Dokumentation erschwert auch die Bewertung, da es auch möglich ist, dass manche Funktionalität mit einer Dokumentation aufzufinden gewesen wäre.\\
Bei den BSI-Standards verhält es sich ähnlich. Sie sollen zwar im Programm enthalten sein, jedoch ist ein Anzeigen nicht möglich, was die Funktionalität der Software stark eingrenzt.\\
\\
Ein Wizard ist zwar vorhanden bietet jedoch keine Erklärungen, die bei der Verwendung des Programms helfen. Dennoch ist es hilfreich, um die einzelnen Schritte in der richtigen Reihenfolge durchzuführen. Grafische Darstellung wie Infrastruktur, Netzpläne und Prozessflüsse fehlen komplett in der Anwendung. Dies liegt mit Sicherheit daran, das SAVe lediglich ein Access Add-In ist. Schnelles Einpflegen von Änderungen ist mit dem Programm sehr leicht durchführbar.\\
\\
Doppelseitige Verlinkungen sind in der Anwendung zwar enthalten, jedoch gibt es keine Möglichkeit einer einseitigen Verlinkung. Eine Vererbung ist generell enthalten, jedoch gibt es auch hier nicht die Möglichkeit einer Auswahl, sondern die Bewertung wird immer vererbt. Die Gruppierung im Programm ist folgendermaßen gestaltet: Informationen, Anwendungen, IT-Systeme, Räume und Kommunikationsverbindungen.\\
\\
Eine Erweiterung der Klassifizierung ist teilweise möglich. Es können neue angelegt werden, jedoch gibt es keine Möglichkeit einer Spezifizierung dieser. Individuelle Beschreibungen sind an jeder Stelle möglich, jedoch fehlt der Zwang eine Beschreibung in speziellen Fällen eintragen zu müssen, wie es im Grundschutztool vorhanden ist. Sicherheitsverstöße werden an keiner Stelle prominent angezeigt, sondern müssen selbst per Hand herausgesucht werden. Eine Detaillierte Bewertung des Sicherheitsstatus war in keiner Ansicht zu finden. Ein Export von Berichten ist zwar möglich, jedoch nur in Form einer CSV-Datei oder einer Excel-Tabelle, die in diesem Fall komplett in Englisch ist, obwohl das Programm komplett auf Deutsch ist. Ein Import vom Grundschutztool soll zwar laut Internetseite möglich sein, jedoch ist eine Option dafür an keiner Stelle zu finden. Die Datenbank kann durch ein Passwort abgesichert werden, jedoch kann jeder, der das Passwort hat, sich die Rolle selbst aussuchen, wodurch eine Rollenvergabe nicht sinnvoll möglich ist. Eine Risikobewertung kann selbständig durchgeführt werden und ist nicht weiter unterstützt.\\
\\
Ein verteiltes Arbeiten wird vom Programm an sich nicht unterstützt, jedoch ist es über Umwege möglich. Eine Rechtevergabe ist möglich, jedoch muss jeder Anwender selbständig darauf achten, dass er die richtige Rolle verwendet. Eine Zertifizierung aufgrund des Programms ist generell nicht möglich. Das Tool wird vom BSI empfohlen, wodurch die Marktpräsenz relativ gut zu bewerten ist. Das Tool ist für alle Nutzergruppen konzipiert. Es möglich eine oder mehrere Consulting-Editionen zu erwerben, dies kostet jedoch pro Lizenz 100 Euro extra. Auch der Kauf einer Version für die Bundeswehr ist auf Anfrage mit Aufpreis möglich. Die aktuelle Version ist vom März 2014. Die vorhergehende Version ist vom September 2012, wodurch auf eine gute Weiterentwicklung zu schließen ist.\\
\\
Bedauerlich jedoch ist, das die Internetseite nicht aktuell ist, was einem einen schlechten Eindruck vermittelt. Insgesamt ist die Software nicht zu empfehlen.